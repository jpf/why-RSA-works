%%%% why-RSA-works/simulation.tex
%%%% Copyright 2012 Peter Franusic.
%%%% All rights reserved.
%%%%

We can simulate RSA messaging using a Lisp interpreter.
Our version of Lisp uses a question mark for the prompt.
Lisp commands may be variable names or compound expressions.
The interpreter reads each command, evaluates it, and prints the result.

Alice must precompute three integers before Bob can send her RSA encrypted messages.
These are \mbox{modulus $n$}, \mbox{encryptor $e$}, and \mbox{decryptor $d$}.
Some examples have been precomputed and are displayed below in decimal format.
Note that \mbox{modulus $n$} has 56 decimal digits.  This turns out to be 186 bits.
A typical RSA modulus has at least 1024 bits, but we use 186 bits here for the sake of brevity.
\begin{quote}
\begin{verbatim}
? n
97397795163266888271167242107545263613895906874319587249
? e
10306926753200670273346978999454444249925952109333797079
? d
46445936998769783647957439537275126296124161350172130481
\end{verbatim}
\end{quote}

A typical RSA message is an AES-128 session key.
This is a random 128-bit integer used by the AES algorithm to encrypt a high bandwidth session.
Here $m$ has 39 decimal digits, or 128 bits.
Normally $m$ would be padded with extra random bits to make it about the same size as $n$,
but we've omitted padding steps in this demonstration for the sake of clarity.
\begin{quote}
\begin{verbatim}
? m
325004947599823818213341565111207349415
\end{verbatim}
\end{quote}

Bob commands his Lisp interpreter to compute \mbox{ciphertext $c$}.
Lisp computes the modex function with inputs $m$, $e$, $n$, 
assigns this value to the variable $c$, then prints the result.
Note that, whereas $m$ has only 39 digis, \mbox{ciphertext $c$} has 56 digits,
the same as $n$.
\begin{quote}
\begin{verbatim}
? (setf c (modex m e n))
65406940630722215589598713946252700262213080283568050086
\end{verbatim}
\end{quote}

Alice commands her Lisp interpreter to compute \mbox{output $y$}.
Lisp computes the modex function with inputs $c$, $d$, $n$, 
assigns this value to the variable $y$, then prints the result.
Note that $y$ is identical to $m$. \emph{Magic!}
\begin{quote}
\begin{verbatim}
? (setf y (modex c d n))
325004947599823818213341565111207349415
\end{verbatim}
\end{quote}

