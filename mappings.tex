%%%% why-RSA-works/mappings.tex
%%%% Copyright 2012 Peter Franusic.
%%%% All rights reserved.
%%%%

%% The goal of this section is to present RSA encryption and decrypttion as two mappings
%% and try to explain why this works.
%% Leverage the modex-33 table to provide a concrete example.
%% Perhaps offer a theorem which asserts: If $e$ is relatively prime to $\lambda$, 
%% then column $e$ will contain all elements of $Z_n$.

%% Introduce the two mappings $m \to m^3$ and $c \to c^7$.
RSA encryption maps an integer $m$ to an integer $m^e$.
Likewise, RSA decryption maps an integer $c$ to an integer $c^d$.
This is demonstrated in Table \ref{modex-33-cols}, where $e=3$ and $d=7$.
In the encryption procedure,
every element in the $m$ column maps to a unique element in the $m^3$ column.
And in the decryption procedure,
every element in the $c$ column maps to a unique element in the $c^7$ column.

%%%% modex-33 columns and arrows
\begin{table}[!h]
  \begin{center}
    %%%% why-RSA-works/modex-33-cols.tex
%%%% Copyright 2012 Pete Franusic.
%%%% All rights reserved.
%%%%
\begin{footnotesize}
\begin{tabular}{ccccccccc}
   $m$  & & $m^3$ & \phantom{XXXXX} & 
   $c$  & & $c^7$ & \phantom{XXXXX} & $y$  \\
        & &       & &       & &       & &       \\
    0   & &   0   & &   0   & &   0   & &   0   \\
    1   & &   1   & &   1   & &   1   & &   1   \\
    2   & &   8   & &   2   & &  29   & &   2   \\
    3   & &  27   & &   3   & &   9   & &   3   \\
    4   & &  31   & &   4   & &  16   & &   4   \\
    5   & &  26   & &   5   & &  14   & &   5   \\
    6   & &  18   & &   6   & &  30   & &   6   \\
    7   & &  13   & &   7   & &  28   & &   7   \\
    8   & &  17   & &   8   & &   2   & &   8   \\
    9   & &   3   & &   9   & &  15   & &   9   \\
   10   & &  10   & &  10   & &  10   & &  10   \\
   11   & &  11   & &  11   & &  11   & &  11   \\
   12   & &  12   & &  12   & &  12   & &  12   \\
   13   & &  19   & &  13   & &   7   & &  13   \\
   14   & &   5   & &  14   & &  20   & &  14   \\
   15   & &   9   & &  15   & &  27   & &  15   \\
   16   & &   4   & &  16   & &  25   & &  16   \\
   17   & &  29   & &  17   & &   8   & &  17   \\
   18   & &  24   & &  18   & &   6   & &  18   \\
   19   & &  28   & &  19   & &  13   & &  19   \\
   20   & &  14   & &  20   & &  26   & &  20   \\
   21   & &  21   & &  21   & &  21   & &  21   \\
   22   & &  22   & &  22   & &  22   & &  22   \\
   23   & &  23   & &  23   & &  23   & &  23   \\
   24   & &  30   & &  24   & &  18   & &  24   \\
   25   & &  16   & &  25   & &  31   & &  25   \\
   26   & &  20   & &  26   & &   5   & &  26   \\
   27   & &  15   & &  27   & &   3   & &  27   \\
   28   & &   7   & &  28   & &  19   & &  28   \\
   29   & &   2   & &  29   & &  17   & &  29   \\
   30   & &   6   & &  30   & &  24   & &  30   \\
   31   & &  25   & &  31   & &   4   & &  31   \\
   32   & &  32   & &  32   & &  32   & &  32   \\
\end{tabular}
\end{footnotesize}

    \caption{$m \to c \to y \quad (\mathcal{R}_{33})$}
    \label{modex-33-cols}
  \end{center}
\end{table}
%%%% why-RSA-works/modex-33-arrows.tex
%%%% Copyright 2012 Peter Franusic.
%%%% All rights reserved.
%%%%

\vspace{-385pt}
\setlength{\unitlength}{1pt} % default value is 1pt
\begin{picture}(345,375)

% Put box around picture.
%% \put(     0,     0){\line(1,0){345}} % bottom
%% \put(     0,375){\line(1,0){345}} % top
%% \put(     0,     0){\line(0,1){375}} % left
%% \put(345,     0){\line(0,1){375}} % right

% Draw arrows to map m to m^3.
\put( 80,343.5){\vector(1,0){16}} % 0
\put( 80,334.0){\vector(1,0){16}} % 1
\put( 80,324.5){\vector(1,0){16}} % 2
\put( 80,315.0){\vector(1,0){16}} % 3
\put( 80,305.5){\vector(1,0){16}} % 4
\put( 80,296.0){\vector(1,0){16}} % 5
\put( 80,286.5){\vector(1,0){16}} % 6
\put( 80,277.0){\vector(1,0){16}} % 7
\put( 80,267.5){\vector(1,0){16}} % 8
\put( 80,258.0){\vector(1,0){16}} % 9
\put( 80,248.5){\vector(1,0){16}} % 10
\put( 80,239.0){\vector(1,0){16}} % 11
\put( 80,229.5){\vector(1,0){16}} % 12
\put( 80,220.0){\vector(1,0){16}} % 13
\put( 80,210.5){\vector(1,0){16}} % 14
\put( 80,201.0){\vector(1,0){16}} % 15
\put( 80,191.5){\vector(1,0){16}} % 16
\put( 80,182.0){\vector(1,0){16}} % 17
\put( 80,172.5){\vector(1,0){16}} % 18
\put( 80,163.0){\vector(1,0){16}} % 19
\put( 80,153.5){\vector(1,0){16}} % 20
\put( 80,144.0){\vector(1,0){16}} % 21
\put( 80,134.5){\vector(1,0){16}} % 22
\put( 80,125.0){\vector(1,0){16}} % 23
\put( 80,115.5){\vector(1,0){16}} % 24
\put( 80,106.0){\vector(1,0){16}} % 25
\put( 80, 96.5){\vector(1,0){16}} % 26
\put( 80, 87.0){\vector(1,0){16}} % 27
\put( 80, 77.5){\vector(1,0){16}} % 28
\put( 80, 68.0){\vector(1,0){16}} % 29
\put( 80, 58.5){\vector(1,0){16}} % 30
\put( 80, 49.0){\vector(1,0){16}} % 31
\put( 80, 39.5){\vector(1,0){16}} % 32

% Draw arrows to map c to c^3.
\put(180,343.5){\vector(1,0){16}} % 0
\put(180,334.0){\vector(1,0){16}} % 1
\put(180,324.5){\vector(1,0){16}} % 2
\put(180,315.0){\vector(1,0){16}} % 3
\put(180,305.5){\vector(1,0){16}} % 4
\put(180,296.0){\vector(1,0){16}} % 5
\put(180,286.5){\vector(1,0){16}} % 6
\put(180,277.0){\vector(1,0){16}} % 7
\put(180,267.5){\vector(1,0){16}} % 8
\put(180,258.0){\vector(1,0){16}} % 9
\put(180,248.5){\vector(1,0){16}} % 10
\put(180,239.0){\vector(1,0){16}} % 11
\put(180,229.5){\vector(1,0){16}} % 12
\put(180,220.0){\vector(1,0){16}} % 13
\put(180,210.5){\vector(1,0){16}} % 14
\put(180,201.0){\vector(1,0){16}} % 15
\put(180,191.5){\vector(1,0){16}} % 16
\put(180,182.0){\vector(1,0){16}} % 17
\put(180,172.5){\vector(1,0){16}} % 18
\put(180,163.0){\vector(1,0){16}} % 19
\put(180,153.5){\vector(1,0){16}} % 20
\put(180,144.0){\vector(1,0){16}} % 21
\put(180,134.5){\vector(1,0){16}} % 22
\put(180,125.0){\vector(1,0){16}} % 23
\put(180,115.5){\vector(1,0){16}} % 24
\put(180,106.0){\vector(1,0){16}} % 25
\put(180, 96.5){\vector(1,0){16}} % 26
\put(180, 87.0){\vector(1,0){16}} % 27
\put(180, 77.5){\vector(1,0){16}} % 28
\put(180, 68.0){\vector(1,0){16}} % 29
\put(180, 58.5){\vector(1,0){16}} % 30
\put(180, 49.0){\vector(1,0){16}} % 31
\put(180, 39.5){\vector(1,0){16}} % 32

% Draw 3 lines for m^3 to c.
\put(114, 220){\line(1,0){23}} % upper horz.
\put(137, 163){\line(0,1){57}} % vertical
\put(137, 163){\vector(1,0){26}} % lower horz.

% Draw 3 lines for c^3 to y.
\put(211, 163){\line(1,0){23}} % lower horz.
\put(234, 163){\line(0,1){57}} % vertical
\put(234, 220){\vector(1,0){26}} % upper horz.

% The End
\end{picture}


% Give an example of $m \to y$.
The table also illustrates an example.
First $m=13$ is mapped to $m^3=19$ which then becomes $c$.
Subsequently $c=19$ is mapped to $c^7=13$ which then becomes $y$.
The final result, $y=13$, is identical to the original, $m=13$.
This is how RSA works, no matter which $m$ one starts with, no matter how large $n$ is.
The final $y$ will always be identical to the original $m$.

%% Here's a question we haven't really answered:
%% Why are the elements in columns $e$ and $d$ arranged the way they are?
%% We know that $(m^e)^d=m$ for all $m$ by way of the proof.
%% But how did the elements in columns $e$ and $d$ get in the right order?

\newpage

%% Point out that some $c=m$.
Some of the values in column $m$ are the same as their corresponding value in $m^3$.
For example, $10 \to 10$.  There are 9 of these, out of a total of 33.
That means that 28 percent of the ciphertexts are identical to their plaintext messages.
This would be unacceptable, of course,
except that the percentage is infinitesimal for huge values of $n$.

%% Point out that each column contains every element in $Z_n$.
Columns $m^3$ and $c^7$ each contain every element in $Z_{33}$.
This is a requirement in order to make the two mappings work, so that $y=m$,
Every element in $m$ needs to map to a unique element in $m^3$,
and every element in $c$ must map to a unique element in $c^7$.
If this were not the case,
if some element in $Z_{33}$ was not in column $m^3$,
or if some element in $Z_{33}$ was in column $m^3$ more than once,
then RSA would not work for every $m$ in $Z_{33}$.

%% Point out that some columns \emph{do not} contain every element,
%% and that neither column number shares a prime factor with $\lambda$.
Refer again to Table \ref{modex-33} in the Wallpaper section.
Many of the columns do not contain every element in $Z_{33}$.
These are columns 0, 2, 4, 5, 6, 8, 10, 12, 14, 15, 16, 18, 20, 22, 24, 25, 26, 28, 30, and 32.
That's 20 out of 33 columns, or 61 percent.
In each of these columns, some elements of $Z_{33}$ are missing and some appear more than once.
For example, the element 1 appears more that once in each of these columns.

It turns out that for each of these columns (except column 0),
the column number shares a prime factor with $\lambda=10$.
The prime factors of 10 are 2 and 5.
So if a column number is a multiple of 2 or a multiple of 5,
the column itself won't contain every element in $Z_{33}$.
Notice that neither 3 nor 7 shares a prime factor with $\lambda$.

%% Define the greatest common divisor function and give an example.
We can use the \emph{greatest common divisor} function ($\gcd$)
to determine if two integers share a prime factor.
When the two integers are small we can compute the greatest common divisor by hand.
As an example we compute the gcd of 6468 and 7560 by hand.
First we list the prime factors of 6468 and the prime factors of 7560.
Then we list the prime factors that are common to both.
Finally, we compute the product of these common factors.
This gives us the gcd.
\begin{center}
\begin{tabular}{lcl}
  Factors of 6468   &=&  $2 \cdot 2 \cdot 3 \cdot 7 \cdot 7 \cdot 11$ \\
  Factors of 7560   &=&  $2 \cdot 2 \cdot 2 \cdot 3 \cdot 3 \cdot 3 \cdot 5 \cdot 7$ \\
  Common to both    &=&  $2 \cdot 2 \cdot 3 \cdot 7$ \\
  $\gcd(6468,7560)$ &=&  84
\end{tabular}
\end{center}

Most implementations of the gcd function return 1 if no prime factors are shared.
Therefore, if the $\gcd$ function returns anything greater than 1,
we are assured that the two numbers share one or more prime factors.
RSA uses the $\gcd$ function during key generation
in order to select an encryptor $e$ that shares no prime factors with $\lambda$.

%% Point out that this pair of column numbers meets the multiples-plus-one condition.
%% Finally, we note that $e=3$ and $d=7$ meet the multiples-plus-one condition.
%% That is, $ed=k\lambda+1$ where $\lambda=10$.
%% \[  ed = (3)(7) = 21 = 2 \cdot 10 + 1  \]

