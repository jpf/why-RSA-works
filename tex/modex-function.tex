%%%% why-RSA-works/modex-function.tex
%%%% Copyright 2012 Peter Franusic.
%%%% All rights reserved.
%%%%

The term \emph{modex} is a contraction of modular exponentiation.
The modex function performs exponentiation in the ring $\mathcal{R}_n$.
It performs the equivalent of a series of $\otimes$ operations.

The RSA cryptosystem in Figure \ref{block-diagram} uses two modex functions: 
one in the transmitter and the other in the receiver.
Both modex functions have three inputs and one output.
We specify the output equation for each.

\paragraph{Receiver output equation:}
The receiver's modex function takes the inputs $c,n,d$ and computes the output $y$.
The modex output $y$ is the equivalent of $d$ copies of $c$ multiplied together using 
the $\otimes$ operator in the ring $\mathcal{R}_n$.
\begin{equation} \label{eq:rx-out}
  y = c^d
\end{equation}

\paragraph{Transmitter output equation:}
The transmitter takes the inputs $m,n,e$ and computes the output $c$.
The modex output $c$ is the equivalent of $e$ copies of $m$ multiplied together using 
the $\otimes$ operator in the ring $\mathcal{R}_n$.
\begin{equation} \label{eq:tx-out}
  c = m^e 
\end{equation}

\vspace{4ex}

The modex function doesn't actually multiply $e$ copies of $m$ in order to compute $m^e$.
This would take eons for huge values of $e$.
Instead, modex actually uses a method called \emph{square-and-multiply}.
A register $r$ is first initialized to $m$.
Then it's repeatedly squared $(r \otimes r)$ and multiplied $(r \otimes m)$
depending on the number of bits in $e$ and the value of each bit.
For example, if $e$ is 1024 bits long, there'll be 1023 squares and about 512 multiplies.
A lot less than $2^{1024}$ multiplies.

The modex function uses the $\otimes$ operator in the ring $\mathcal{R}_n$.
Recall that the $\otimes$ operator takes two integers, multiplies them,
then subtracts some multiple of $n$ so that the result is in $Z_n$.
That is, $a \otimes b = ab - kn$.
The subtraction step is called \emph{reduction} and may be implemented
by taking the remainder of a division.
The product $ab$ is divided by $n$, the quotient is $k$, and the remainder is $ab-kn$.
But division is time consuming, and
most modex implementations do not use division for the reduction step.
Instead, they use a faster method called \emph{Montgomery reduction},
which replaces slow divisions with fast truncations.

The modex function can be very time-consuming to compute.
Square-and-multiply and Montgomery reduction are two \emph{speed-ups}
that are used to shorten the compute time.  There are others.
The enumeration and details of these speed-ups are outside the scope of this paper,
but they are well-documented in the literature.
\cite{Koc}\cite{Schneier}\cite{HAC}

