%%%% why-RSA-works/hard-problems.tex
%%%% Copyright 2012 Peter Franusic.
%%%% All rights reserved.
%%%%

%%%% Note that RSA trades the problem of key distribution for different problems
%%%% that are considered \emph{hard}.

The security of RSA is based on several hard problems.
The most prominent of these is \emph{integer factorization}.
The problem is to write an algorithm that computes the prime factors of some huge integer $n$
and does it using a small number of computing operations.
An algorithm is ``fast'' if it requires only a few operations to complete the solution.
It is a hard problem to write an algorithm that is fast enough 
to factor a 1024-bit RSA modulus within any reasonable amount of time.

Four factoring algorithms are graphed in Figure \ref{factor-ops}.
They are, from slowest to fastest: Trial Division (TD), the Quadratic Sieve (QS),
the Number Field Sieve (NFS), and Peter Shor's algorithm for quantum computers.\cite{Shor}
The graph plots the number of operations required to factor some modulus $n$.
For example, it will take roughly $10^{12}$ operations 
to factor a 768-bit modulus using the NFS algorithm.
This is about 1500 years on a single core 2.2 GHz AMD Opteron processor with 2 GB RAM.\cite{RSA-768}
%%%% Trial Division (TD): $\mathcal{O}(\sqrt{N})$ operations.
%%%% Quadratic Sieve (QS): $\mathcal{O}(e^{(\ln N)^{1/2}(\ln (\ln N))^{1/2}})$ operations.
%%%% Number Field Sieve (NFS): $\mathcal{O}(e^{(\ln N)^{1/3}(\ln (\ln N))^{2/3}})$ operations.
%%%% Quantum algorithm (Shor): $\mathcal{O}((\ln N)^3)$ operations.

%%%% Graph of factoring times
%%%% Present three graphs: TD, QS, NFS.
%%%% Bits on linear scale, operations on log scale.

\begin{figure}[h]
\vspace{4ex}
\begin{center}
%%%% why-RSA-works/factor-ops.tex
%%%% Copyright 2012 Peter Franusic.
%%%% All rights reserved.
%%%%
%% This is LaTeX source code for a figure that contains four curves.
%% The curves are specified by LaTeX and Lisp expressions shown below.
%% The labels of the curves are TD, QS, NFS, and Shor.
%% TD = Trial Division factoring algorithm.
%% QS = Quadratic Sieve factoring algorithm.
%% NFS = Number Field Sieve factoring algorithm.
%% Shor = Peter Shor's factoring algorithm for quantum computers.
%% The curves are overlayed on a 64 by 30 grid pattern.
%% The x-axis has lines every 4 grids, with labels {128,256,384,...,1204}.
%% The y-axis has lines every 3 grids, with labels {6,12,18,24,30}.
%%
%% In the Lisp code below, the expt function will accept integer exponents
%% greater than 128 but not floating-point exponents.
%% E.g., (expt 2 129) returns 680564733841876926926749214863536422912,
%% but (expt 2 129.0) causes an error message to be printed.
%% 
%% TD curve:
%%   $y = \log \left( \sqrt{2^x} \right)$
%%   (setf y (log (sqrt (expt 2 x)) 10))
%% 
%% QS curve:
%%   $y = \log \left( e^{\left( \left( \ln \; 2^{x} \right)^{\frac{1}{2}}\; \cdot \; 
%%        \left( \ln \; \left( \ln \; 2^{x} \right) \right)^{\frac{1}{2}} \right)} \right)$
%%   (setf y (log (exp (* (expt (log (expt 2 x)) 1/2) (expt (log (log (expt 2 x))) 1/2))) 10))
%% 
%% NFS curve:
%%   $y = \log \left( e^{\left( \left( \ln \; 2^{x} \right)^{\frac{1}{3}}\; \cdot \; 
%%        \left( \ln \; \left( \ln \; 2^{x} \right) \right)^{\frac{2}{3}} \right)} \right)$
%%   (setf y (log (exp (* (expt (log (expt 2 x)) 1/3) (expt (log (log (expt 2 x))) 2/3))) 10))
%% 
%% Shor curve:
%%   $y = \log \left( \left( \ln \left( 2^{x} \right) \right)^{3} \right)$
%%   (setf y (log (expt (log (expt 2 x)) 3) 10))
%% 

\setlength{\unitlength}{1.6mm}
\begin{picture}(64,30) 
\linethickness{0.075mm} 

%% grid pattern
%% \multiput (x,y) (dx,dy) {n} {object}
\multiput (0,0) (8,0) {9} {\line(0,1){30}} % x divisions
\multiput (0,0) (0,6) {6} {\line(1,0){64}} % y divisions

%% y-axis labels
%% 6 12 18 24 30 
\put (-2.5, 29.5){\scriptsize\textsf{30}}
\put (-2.5, 23.5){\scriptsize\textsf{24}}
\put (-2.5, 17.5){\scriptsize\textsf{18}}
\put (-2.5, 11.5){\scriptsize\textsf{12}}
\put (-2.2,  5.5){\scriptsize\textsf{ 6}}

%% x-axis labels
%% 128 256 384 512 640 768 896 1024
\put( 6.8,-2.0){\scriptsize\textsf{128}}
\put(14.8,-2.0){\scriptsize\textsf{256}}
\put(22.8,-2.0){\scriptsize\textsf{384}}
\put(30.8,-2.0){\scriptsize\textsf{512}}
\put(38.8,-2.0){\scriptsize\textsf{640}}
\put(46.8,-2.0){\scriptsize\textsf{768}}
\put(54.8,-2.0){\scriptsize\textsf{896}}
\put(62.0,-2.0){\scriptsize\textsf{1024}}

\thicklines

%% TD curve
%% \qbezier (start-x,start-y) (pull-x,pull-y) (stop-x,stop-y)
\put (7.0, 26.0) {\textsf{TD}}
\qbezier (0.00, 0.00) (6.25, 15.00) (12.50, 30.00)

%% QS curve
\put (42.5, 25.0) {\textsf{QS}}
\qbezier (0.00, 0.00) (7.00, 15.00) (64.00,29.65)

% NFS curve
\put (50.0, 13.25) {\textsf{NFS}}
\qbezier (0.00,0.00) (5.50, 9.00) (64.00,13.58)

%% Shor curve
\put (51.0, 9.25) {\textsf{Shor}}
\qbezier (0.00,0.00) ( 0.50,4.50) ( 8.00,5.84)
\qbezier (8.00,5.84) (24.00,8.00) (64.00,8.55)

\end{picture}

\vspace{2ex}
\caption{$\log_{10}$ operations per $\log_2 n$}
\label{factor-ops}
\end{center}
\end{figure}

An RSA cryptosystem can be broken if the modulus can be factored.
That is, if Eve can factor $n$ into $p$ and $q$, she can easily compute $d$.
She first computes the Carmichael function value $\lambda=\lcm(p-1,q-1)$.
Then she computes $d$ such that $ed=k\lambda + 1$.
Trouble is, factoring a huge integer takes a \emph{very} long time.

Eve can try to solve the RSA problem\cite{RSA-problem} and 
compute the $e^{th}$ root of $m^e$,
i.e., compute $m = \sqrt[e]{m^e}$.
But computing roots takes just as long as factoring.
There are other algorithms that can theoretically break RSA
but they're all just as slow as integer factorization.
The point is that Eve will not be able break an RSA cryptosystem with a huge modulus 
in any reasonable amount of time.

