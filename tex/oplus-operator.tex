%%%% why-RSA-works/oplus-operator.tex
%%%% Copyright 2012 Peter Franusic.
%%%% All rights reserved.
%%%%

When $n$ is small, the $\oplus$ operator can be specified using a table.
Table \ref{oplus-15} specifies the $\oplus$ operator for the ring $\mathcal{R}_{15}$.
The table is a 15 by 15 block of integers.  There are 15 rows and 15 columns.
Recall that rows run left and right, columns run up and down.
Row numbers are specified by the extra column along the left side of the block.  
Column numbers are specified by the extra row along the top of the block.
Note the diagonal stripe pattern that is visible in the table.

\vspace{2ex}
%%%% oplus-15 table
\begin{table}[!ht]
  \begin{center}
    %%%% why-RSA-works/oplus-15.tex
%%%% Copyright 2012 Peter Franusic.
%%%% All rights reserved.
%%%%
\begin{footnotesize}
\begin{tabular}
    {c@{ }c@{ }c@{ }c@{ }c@{ }c@{ }c@{ }c@{ }c@{ }c@{ }c@{ }c@{ }c@{ }c@{ }c@{ }c@{ }c}
        & \phantom{X}
             &  0 &  1 &  2 &  3 &  4 &  5 &  6 &  7 &  8 &  9 & 10 & 11 & 12 & 13 & 14 \\
        &    &    &    &    &    &    &    &    &    &    &    &    &    &    &    &    \\
    0   &    &  0 &  1 &  2 &  3 &  4 &  5 &  6 &  7 &  8 &  9 & 10 & 11 & 12 & 13 & 14 \\
    1   &    &  1 &  2 &  3 &  4 &  5 &  6 &  7 &  8 &  9 & 10 & 11 & 12 & 13 & 14 &  0 \\
    2   &    &  2 &  3 &  4 &  5 &  6 &  7 &  8 &  9 & 10 & 11 & 12 & 13 & 14 &  0 &  1 \\
    3   &    &  3 &  4 &  5 &  6 &  7 &  8 &  9 & 10 & 11 & 12 & 13 & 14 &  0 &  1 &  2 \\
    4   &    &  4 &  5 &  6 &  7 &  8 &  9 & 10 & 11 & 12 & 13 & 14 &  0 &  1 &  2 &  3 \\
    5   &    &  5 &  6 &  7 &  8 &  9 & 10 & 11 & 12 & 13 & 14 &  0 &  1 &  2 &  3 &  4 \\
    6   &    &  6 &  7 &  8 &  9 & 10 & 11 & 12 & 13 & 14 &  0 &  1 &  2 &  3 &  4 &  5 \\
    7   &    &  7 &  8 &  9 & 10 & 11 & 12 & 13 & 14 &  0 &  1 &  2 &  3 &  4 &  5 &  6 \\
    8   &    &  8 &  9 & 10 & 11 & 12 & 13 & 14 &  0 &  1 &  2 &  3 &  4 &  5 &  6 &  7 \\
    9   &    &  9 & 10 & 11 & 12 & 13 & 14 &  0 &  1 &  2 &  3 &  4 &  5 &  6 &  7 &  8 \\
   10   &    & 10 & 11 & 12 & 13 & 14 &  0 &  1 &  2 &  3 &  4 &  5 &  6 &  7 &  8 &  9 \\
   11   &    & 11 & 12 & 13 & 14 &  0 &  1 &  2 &  3 &  4 &  5 &  6 &  7 &  8 &  9 & 10 \\
   12   &    & 12 & 13 & 14 &  0 &  1 &  2 &  3 &  4 &  5 &  6 &  7 &  8 &  9 & 10 & 11 \\
   13   &    & 13 & 14 &  0 &  1 &  2 &  3 &  4 &  5 &  6 &  7 &  8 &  9 & 10 & 11 & 12 \\
   14   &    & 14 &  0 &  1 &  2 &  3 &  4 &  5 &  6 &  7 &  8 &  9 & 10 & 11 & 12 & 13 \\
\end{tabular}
\end{footnotesize}

    \caption{$a \oplus b \quad (\mathcal{R}_{15})$}
    \label{oplus-15}
  \end{center}
\end{table}

Table \ref{oplus-15} specifies the value of $a \oplus b$ for every possible pair of $a$ and $b$.
For example, let $a=10$ and $b=12$.
The value of $10 \oplus 12$ is specified at the intersection of row $10$ and column $12$.
This value is $7$.  Therefore $10 \oplus 12 = 7$.

Notice that every element in the table is in the set $Z_{15}$.
This demonstrates the \emph{additive closure} property of rings.
The additive closure property states that for every pair of elements $a$ and $b$ in $Z_n$,
the sum $a \oplus b$ is also an element in $Z_n$.
\[ a \oplus b \in Z_n \]

The value of $a \oplus b$ can also be specified using a rule.
To compute $10 \oplus 12$ we first calculate $10 + 12$ to get 22.
Since 22 is not an element in $Z_{15}$ we subtract the modulus 15, i.e. $22 - 15 = 7$.
Since $7 \in Z_{15}$ we stop and 7 is our final result.

In general, the $\oplus$ operator takes two integers $a$ and $b$, 
adds them together using normal addition, 
then subtracts some multiple of $n$ such that the final value is in $Z_n$.
The term $kn$ signifies some multiple of $n$. That is, $kn=0n,1n,2n,3n,\ldots$
We simply use whichever $kn$ works in order to get closure, where $a \oplus b \in Z_n$.
\[  a \oplus b = a + b - kn  \]

