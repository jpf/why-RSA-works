%%%% why-RSA-works/conclusions.tex
%%%% Copyright 2012 Peter Franusic.
%%%% All rights reserved.
%%%%

So why does RSA work?
Why is it that we can take some message $m$,
run it through two modex operations, and come out with the same $m$?
There are several reasons.
First of all, RSA computations are done in a commutative ring
and the multiplicative association property holds in commutative rings.
This property tells us that 
the two exponentiations $(m^e)^d$ are the same as the one exponentiation $m^{ed}$.

A second reason is that exponents $e$ and $d$ are chosen
such that they satisfy the multiples-plus-one condition $ed = k\lambda + 1$.
This insures that $ed$ is one of the identity columns
in the exponential table of ring $\mathcal{R}_n$.

A third reason is that the exponential table contains 
repeating blocks of columns where $m^a=m^{k\lambda+a}$.
This is the wallpaper pattern that we saw in Table \ref{modex-33}.
This pattern is the reason for the multiples-plus-one condition.

Finally, RSA works because it relies on the intractability of the factoring problem.
A huge RSA modulus $n$ cannot be factored expeditiously.
Given that $n$ is the product of two distinct huge random primes,
it is virtually impossible to factor $n$ in any reasonable amount of time,
even if the factoring effort is distributed across thousands of computers.

