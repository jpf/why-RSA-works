%%%% why-RSA-works/exponent-arithmetic.tex
%%%% Copyright 2012 Peter Franusic.
%%%% All rights reserved.
%%%%

RSA uses exponential notation in the ring $\mathcal{R}_n$.
Exponential notation is simply a mathematical shorthand for writing 
a series of multiplications using the $\otimes$ operator.
The multiplicative association property allows us to derive 
two rules for doing arithmetic with exponents.

Consider the set of three equations below.
The left side of the first equation is the expression $m^2 \otimes m^3$.
We can replace the $m^2$ with $(m \otimes m)$.
We can also replace the $m^3$ with $(m \otimes m \otimes m)$.
The right side of the first equation shows this.
The multiplicative association property says that we can 
ignore the parentheses and simply count the number of $m$'s that are multiplied.
There are 5 and we show this in the second equation.
Note that 5 is the sum of 2 plus 3.
So instead of expanding the expression $m^2 \otimes m^3$
we can simply add 2 and 3, as shown in the third equation.
\begin{eqnarray*}
  m^2 \otimes m^3 &=&  (m \otimes m) \otimes (m \otimes m \otimes m)  \\
  &=&  m^5  \\
  &=&  m^{2 + 3}
\end{eqnarray*}

\paragraph{Exponent addition rule:}  In general, when we have an expression of the form 
$m^e \otimes m^d$ in the ring $\mathcal{R}_n$ we can simply add the exponents.
\[  m^e \otimes m^d  =  m^{e + d}  \]

Consider the set of four equations below.
The left side of the first equation is the expression $(m^2)^3$.
This means three copies of $m^2$ are multiplied using the $\otimes$ operator.
The right side of the first equation shows this.
In the second equation, we replace each $m^2$ with $(m \otimes m)$.
The multiplicative association property says that we can 
ignore the parentheses and simply count the number of $m$'s that are multiplied.
There are 6 and we show this in the third equation.
Note that 6 is the product of 2 times 3.
Instead of expanding the expression $(m^2)^3$
we can simply multiply 2 and 3, as shown in the fourth equation.
\begin{eqnarray*}
  (m^2)^3  &=&  m^2 \otimes m^2 \otimes m^2  \\
  &=&  (m \otimes m) \otimes (m \otimes m) \otimes (m \otimes m)  \\
  &=&  m^6 \\
  &=&  m^{2 \times 3}
\end{eqnarray*}

\paragraph{Exponent multiplication rule:}  In general, when we have an expression of the form 
$(m^e)^d$ in the ring $\mathcal{R}_n$ we can simply multiply the exponents.
\begin{equation} \label{eq:expo-mult}
  (m^e)^d  =  m^{ed} 
\end{equation}

