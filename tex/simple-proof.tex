%%%% why-RSA-works/simple-proof.tex
%%%% Copyright 2012 Peter Franusic.
%%%% All rights reserved.
%%%%

We need to convince ourselves that RSA works under a broad set of conditions.
That is, we need to demonstrate that we can start with any $m \in Z_n$,
perform two modex operations on it, and get $m$ back.
Here's the set of conditions:
\begin{itemize}
\item  two prime integers $p$ and $q$ such that $p \ne q$
\item  the ring $\mathcal{R}_n = (Z_n,\oplus,\otimes)$ where $n=pq$
\item  exponential notation in $\mathcal{R}_n$  (e.g. $m^3 = m \otimes m \otimes m$)
\item  the Carmichael function value $\lambda=\lcm(p-1,q-1)$
\item  two integers $e$ and $d$ such that $ed=k\lambda + 1$
\item  an integer $m$ such that $m \in Z_n$
\end{itemize}

Refer to the RSA cryptosystem of Figure \ref{block-diagram}.
The message $m$ is presented at the input of Bob's transmitter.
The message $y$ is produced at the output of Alice's receiver.
We will demonstrate that $y=m$.

The receiver output equation (\ref{eq:rx-out}) states that $y=c^d$.
This is what we begin our proof with.
In the following steps, we will modify the right side of this equation.
\[  y = c^d  \]

The transmitter output equation (\ref{eq:tx-out}) states that $c=m^e$.
We replace $c$ in the equation above with $(m^e)$.
\[  y = (m^e)^d  \]

The exponent multiplication rule (\ref{eq:expo-mult}) states that $(m^e)^d=m^{ed}$.
We replace $(m^e)^d$ in the equation above with $m^{ed}$.
\[  y = m^{ed}  \]

The multiple-plus-one condition (\ref{eq:inv-pair}) states that $ed=k\lambda + 1$.
We replace $ed$ in the equation above with $k\lambda + 1$.
\[  y = m^{k\lambda + 1}  \]

The Carmichael identity (\ref{eq:carm-id}) states that $m=m^{k\lambda + 1}$.
We replace $m^{k\lambda + 1}$ in the equation above with $m$.
\[  y = m  \]

QED

