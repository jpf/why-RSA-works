%%%% why-RSA-works/intro.tex
%%%% Copyright 2012 Peter Franusic.
%%%% All rights reserved.
%%%%

Arthur C. Clarke once quipped that
``any sufficiently advanced technology is indistinguishable from magic.''
Cryptography is the magic that
transmogrifies a meaningful message into gibberish and then back again.
For thousands of years, military-grade cryptography was the exclusive domain of 
diplomats and generals, partly due to the high cost of keeping secret keys secret.
But around 1975 something happened to change all that: \emph{public-key} cryptography was invented.
Public-key cryptography dramatically reduces the cost of secret key management.

The Rivest-Shamir-Adleman algorithm (RSA) is a well-established computational method 
for public-key cryptography.\cite{RSA-paper}
We offer the reader an understanding of why RSA works.
A simple proof of the RSA identity is developed using an illustrative approach.
Table \ref{modex-33} is particularly revealing.

The scope of the article is limited to understanding the RSA identity.
The discussion therefore omits related topics such as multi-prime RSA, key generation,
conversion of text to integers and integers to text, padding of cleartext messages,
various speed-ups such as Montgomery reduction and the Chinese remainder theorem,
and the latest factoring algorithms.
RSA authentication is not covered because the math is identical to that of RSA encryption.

The presentation differs in several ways from conventional treatments of the RSA algorithm.
The algebraic equations utilize ring notation and equal signs
rather than modular notation and equivalence signs.
Instead of Euler's totient function and Fermat's little theorem,
a proof of the RSA identity employs the Carmichael function
and a corollary from the literature.\cite{ray-attack}

