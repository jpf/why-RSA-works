%%%% why-RSA-works/multiple-plus-one.tex
%%%% Copyright 2012 Peter Franusic.
%%%% All rights reserved.
%%%%

RSA uses two integers as exponents.
One is the encryptor $e$ and the other is the decryptor $d$.
In order for RSA to work, the product $ed$ must satisfy a strict condition.
The condition is that the product $ed$ must have a \emph{multiple-plus-one} form.
The product must be able to be written in the form $k\lambda+1$.
The reason for this condition will become apparent later.
For now, however, we need to understand what the expression $k\lambda+1$ means.

%% This paragraph shall introduce Table \ref{mult-plus-one} below.
Table \ref{mult-plus-one} contains some examples of multiple-plus-one products.
Each product ends in 001.
Each product is a multiple of 1000, plus one.
In the first example, the product 174001 is equal to $174 \cdot 1000 + 1$.

\vspace{2ex}
%%%% multiple-plus-one table
\begin{table}[!ht]
  \begin{small}
    %%%% why-RSA-works/mult-plus-one.tex
%%%% Copyright 2012 Peter Franusic.
%%%% All rights reserved.
%%%%
\begin{tabular}{llll}
$911 \cdot 191 = 174001$ & $931 \cdot 971 = 904001$ & $951 \cdot 551 = 524001$ & $971 \cdot 931 = 904001$ \\
$913 \cdot 977 = 892001$ & $933 \cdot 597 = 557001$ & $953 \cdot 617 = 588001$ & $973 \cdot 37  =  36001$ \\
$917 \cdot 253 = 232001$ & $937 \cdot 873 = 818001$ & $957 \cdot  93 =  89001$ & $977 \cdot 913 = 892001$ \\
$919 \cdot 679 = 624001$ & $939 \cdot 459 = 431001$ & $959 \cdot 439 = 421001$ & $979 \cdot 619 = 606001$ \\
$921 \cdot 481 = 443001$ & $941 \cdot 661 = 622001$ & $961 \cdot 641 = 616001$ & $981 \cdot 421 = 413001$ \\
$923 \cdot 987 = 911001$ & $943 \cdot 807 = 761001$ & $963 \cdot  27 =  26001$ & $983 \cdot 647 = 636001$ \\
$927 \cdot 863 = 800001$ & $947 \cdot 283 = 268001$ & $967 \cdot 303 = 293001$ & $987 \cdot 923 = 911001$ \\
$929 \cdot 169 = 157001$ & $949 \cdot 549 = 521001$ & $969 \cdot 129 = 125001$ & $989 \cdot 909 = 899001$ \\
\end{tabular}

  \end{small}
  \caption{Multiples of 1000, plus one}
  \label{mult-plus-one}
\end{table}
\vspace{2ex}

%% This paragraph shall introduce $\lambda$.
The Greek letter $\lambda$ (pronounced \textsf{LAM duh})
is specified in the RSA literature.\cite{RSA-standard}
We use $\lambda$ here as an integer constant.
It typically has a huge value, almost as large as modulus $n$.
In the context of Table \ref{mult-plus-one} it has a small value, $\lambda=1000$.
The products can therefore be written like this:
\begin{eqnarray*}
  911 \cdot 191  &=&  174 \lambda + 1 \\
  913 \cdot 977  &=&  892 \lambda + 1 \\
  917 \cdot 253  &=&  232 \lambda + 1 \\
  & \vdots &
\end{eqnarray*}

%% This paragraph shall explain that $k$ is some unspecified positive integer.
The products in the table can be written in the form $k\lambda+1$.
The symbol $k$ signifies some positive integer.
Its value is not important.
The term $k\lambda$ simply means \emph{some integer multiple of $\lambda$}.
This meaning of $k$ allows the multiple-plus-one condition to be stated succinctly.

%% This paragraph shall formally state the condition.
\paragraph{Multiple-plus-one condition:}
Given positive integers $e$, $d$, and $\lambda$,
the product $ed$ shall be an integer multiple of $\lambda$, plus one.
\begin{equation} \label{eq:inv-pair}
  ed = k\lambda + 1
\end{equation}

